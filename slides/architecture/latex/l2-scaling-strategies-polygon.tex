% !TeX document-id = {38ad272c-6b95-4178-b409-8ae7f3766667}
% !TeX spellcheck = en_US
% !TeX root = ../../build/architecture.tex
% !TeX TXS-program:compile = txs:///xelatex/[--shell-escape]


\renewcommand{\mytitle}{Polygon L2 Scaling Strategies}
\ifZEROSEC \fi
\ifSEC \section{\mytitle{}}\fi
\ifSUBSEC \subsection{\mytitle{}}\fi
\ifSUBSUBSEC \subsubsection{\mytitle{}}\fi


% \begin{frame}{L2 Scaling and Strategies}
% Review the road to scalability and the
% L2 scaling strategies at the concepts.
% \end{frame}




\begin{frame}{L2 Design for the Polygon zkEVM}
\begin{enumerate}[a)]
\item How users send L2 transactions and who receives them?
  \begin{itemize}
  \item The zkEVM uses \textbf{unicast} to let user send their transaction (calls to an RPC).
  \item The zkEVM also enables posting L2 transactions via a method in a \textbf{smart contract}
  as an anti-censorship measure (called "forced batches").
  \end{itemize}
\item How L2 transactions are made publicly available (if so)?
  \begin{itemize}
  \item The zkEVM is a \textbf{rollup}, the L2 data is available in L1.
  \end{itemize}
\item Who processes the L2 transactions and how, and, when
it is publicly considered that a new state is correctly computed?
  \begin{itemize}
  \item In the zkEVM, currently, there is a \textbf{centralized aggregator node} that proves the
 processing of the L2 transactions.
  \item However, this node cannot cheat because there is a \textbf{succinct computation verification}
  (using zero-knowledge technology).
  \end{itemize}
\item What type of applications the L2 supports? simple or rich processing?
  \begin{itemize}
  \item zkEVM is rich processing since it is an \textbf{EVM}.
  \end{itemize}
\end{enumerate}
\end{frame}